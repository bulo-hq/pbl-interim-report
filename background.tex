\chapter{背景・目的・手法}

\section{背景}
\subsection{プロジェクトの立ち上げ}
世の中にはニーズを十分に満たしていないシステムが存在する.これは,開発者が作るものとユーザが求めるものとの乖離が原因の1つであると考えられる.この問題を解決するためには,ユーザを理解してシステムを開発する必要があることから,開発者が現場に赴き,調査をして,ユーザから直接学ぶべきであると考えた.そこで「使ってもらって学ぶフィールド指向システムデザイン」を理念とするプロジェクトが始まった.

\subsection{プロジェクトの方針}
本プロジェクトは,例年実際に現場に赴くフィールド調査とアジャイル開発手法の 1 つである,
スクラム手法 \cite{scrum}を採用している.フィールド調査ではユーザの思考や行動などの,現場に行かないと分からないことを知ることができる.また,スクラムは,アジャイル開発の中でも少人数のチームで開発を行い,スプリントと呼ばれる固定の短い時間に区切って作業を進めるものである\cite{scrum}.これはユーザのフィードバックを繰り返し受けて,改善する機会を何度も得ることができるため,本プロジェクトの「使ってもらって学ぶフィールド指向システムデザイン」という理念と合致する.以上のことから,今年度もフィールド調査とスクラム手法を採用することにした.

\subsection{交通グループ}\label{sec:gaiyou}
発車予定時刻ギリギリにバス停についたときに,バスが今どこにいるのかがわからないことで, バスがもう行ってしまったのか, まだ来ていないのかがわからないという現状がある.また,現在存在する地図アプリや,乗換案内のアプリ,バス会社が提供しているバスロケーションシステムによって,バスの運行に関する情報を手に入れることができる.しかし,それらの情報は大量で整理されておらず, 欲しい情報に辿り着くのが難しい.先述の問題を既存のアプリやシステムで解決することが困難であることから,本グループでは,バス利用者がストレスを抱えずにバスを利用できるようなアプリ開発を行うことにした.

\bunseki{大津武琉}
